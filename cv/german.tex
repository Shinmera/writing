\documentclass[12pt,a4paper]{article}
\usepackage[utf8]{luainputenc}
\usepackage{fontspec}
\usepackage[top=1.5cm, bottom=2.5cm, left=1cm, right=1cm]{geometry}
\usepackage{enumitem}
\usepackage{setspace}

\title{Lebenslauf}
\author{Nicolas Hafner}
\date{März 2015}

\defaultfontfeatures{Ligatures=TeX}
\setmainfont{Equity Text A}
\setsansfont{Concourse T3}
\setmonofont{Triplicate T4}
\newfontfamily\titles{Concourse C3}

\newcommand{\toptitle}[1]{%
  \vskip 0.5cm
  \par\hbox{\titles\selectfont\Large #1\strut}\hrule
  \vskip 0.5cm
}

\newenvironment{topsection}[1]{
  \toptitle{#1}
}{

}

\newcommand{\itemtitle}[2]{%
  \vskip 0.2cm
  {\Large #1} \hfill\hfill {\textbf #2}%
}

\newenvironment{itemsection}[2]{
  \itemtitle{#1}{#2}
  \begin{itemize}[itemsep=0em]
}{
  \end{itemize}
}

\begin{document}

\begin{center}
  \doublespacing
  \titles\selectfont
  {\Huge Nicolas Hafner} \\
  Nürenbergstrasse 17B, 8037 Zürich, Schweiz \hskip 0.5cm +41 76 579 90 38\\
  shinmera@tymoon.eu \hskip 1cm https://shinmera.com
\end{center}

\begin{topsection}{Ausbildung}
  \begin{itemsection}{Gymnasium Kantonsschule Oerlikon}{8.2007 - 7.2012}
  \item Erfolgreicher Matura-Abschluss.
  \end{itemsection}
  
  \begin{itemsection}{Bachelor ETH Zürich}{9.2013 - Heute}
  \item Eintritt in Rechnergestütze-Wissenschaften
  \item Im September 2014 Wechsel zur Informatik
  \end{itemsection}
\end{topsection}  


\begin{topsection}{Arbeitserfahrung}
  \begin{itemsection}{PLANTA-GmbH Schweiz}{9.2012 - 6.2013}
  \item PLANTA Projektmanagement \& PLANTA Customizer Ausbildung
  \item Entwicklung von Deploymentverfahren mit Python
  \end{itemsection}
\end{topsection}

\begin{topsection}{Weitere Kenntnisse}
  \begin{itemsection}{Open-Source Projekte}{\hfill}
  \item Diverse Projekte in Java und Common Lisp veröffentlicht auf Github \hfill (https://github.shinmera.com)
  \end{itemsection}

  \begin{itemsection}{Künstlerisches}{\hfill}
  \item Regelmässige Sketche, Zeichnungen und Bilder \hfill (https://tumblr.shinmera.com)
  \end{itemsection}

  \begin{itemsection}{Sprachen}{\hfill}
  \item Muttersprachen Deutsch und Schweizerdeutsch
  \item Grundausbildung in Französisch
  \item Fliessendes Englisch
  \item Rudimentäre Japanisch Kenntnisse
  \end{itemsection}
\end{topsection}
\end{document}

%%% Local Variables:
%%% mode: latex
%%% TeX-master: t
%%% TeX-engine: luatex
%%% End:
